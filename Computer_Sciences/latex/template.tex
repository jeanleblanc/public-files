\documentclass{report}

\input{preamble}
\input{macros}
\input{letterfonts}

\title{\Huge{Class}\\Sub}
\author{\huge{Name}}
\date{}

\begin{document}

\maketitle
\newpage% or \cleardoublepage
% \pdfbookmark[<level>]{<title>}{<dest>}
\pdfbookmark[section]{\contentsname}{toc}
\tableofcontents
\pagebreak


\begin{verbatim}
	beg = \begin{document} ... \end{document}
	mk = \(\)
	sub = base_index
	sup = base^{:exponent}
	sr = ^2
	cb = ^3
	compl = ^{c}
	frac = \frac{numerator}{denominator}
	bar = \overline{variable}
	hat = \hat{variable}
	vec = \vec{variable}


\end{verbatim}



\dfn{title}{content} = définition

\sol = solution

\nt = note

\begin{note}     
\end{note}   

\clm{title}{}{content} = claim

\ex{title}{content} = example

\thm{title}{content} = thm

\begin{myproof} = proof
\end{myproof}

Utilisation de tikzpicture

\cor{title}{content} = corrollary

\mlenma{title}{content} : lemme

\mprop{title}{content} : proposition

numéroter les equations : \begin{enumerate} \item\label{n:1}\end{enumerate}

\qs{}{content}


\begin{algorithm}
\KwIn{input}
\KwOut{output}
\SetAlgoLined
\SetNoFillComment
\tcc{This is a comment}
\tcp*{}
\uIf{content}{}
\Else{content}
\ForEach{}{}
\For{}{}
\While{}{}
\Return something
\caption{title of the algo}
\end{algorithm}



\end{document}
